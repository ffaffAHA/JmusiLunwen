% !TeX encoding = UTF-8

% 载入 SJTUThesis 模版
\documentclass[type=bachelor,zihao=5,oneside,openany]{sjtuthesis}
% 选项
%   type=[doctor|master|bachelor],     % 可选(默认:master),论文类型
%   zihao=[-4|5],                      % 可选(默认:-4),正文字号大小
%   lang=[zh|en|de|ja],                % 可选(默认:zh),论文的主要语言
%   review,                            % 可选(默认:关闭),盲审模式
%   [twoside|oneside],                 % 可选(默认:twoside),双页或单页边距模式
%   [openright|openany],               % 可选(默认:openright),奇数页或任意页开始新章
%   math-style=[ISO|TeX],              % 可选 (默认:ISO),数学符号样式

% 论文基本配置,加载宏包等全局配置
% !TEX root = ./main.tex

\sjtusetup{
  %
  %******************************
  % 注意:
  %   1. 配置里面不要出现空行
  %   2. 不需要的配置信息可以删除
  %******************************
  %
  % 信息录入
  %
  info = {%
    %
    % 标题标标标标标标标标标标标标标标标题题题题题题题题题题题题题题题题题题
    %
    zh / title           = {\LaTeX{} 模板示例文档},
    en / title           = {A Sample Document for \LaTeX-based SJTU Thesis Template},
    %
    % 标题页标题
    %   可使用“\\”命令手动控制换行
    %
    % zh / display-title   = {上海交通大学学位论文\\ \LaTeX{} 模板示例文档},
    % en / display-title   = {A Sample Document \\ for \LaTeX-based SJTU Thesis Template},
    %
    % 关键词关键词关键词关键词关键词关键词关键词关键词关键词关键词关键词关键词关键词关键词
    zh / keywords        = {上海交大, 饮水思源, 爱国荣校},
    en / keywords        = {SJTU, Major aster thesis, XeTeX/LaTeX template},
    %
    % 姓名姓名姓名姓名姓名姓名姓名姓名姓名姓名姓名姓名姓名姓名姓名姓名姓名姓名姓名姓名姓名姓名
    %
    zh / author          = {娄\quad{}金\quad{}尚},
    en / author          = {Lou~jin~shang},
    %
    % 指导教师
    %
    zh / supervisor      = {某某教授},
    en / supervisor      = {Prof. Mou Mou},
    %
    % 副指导教师
    %
    % assoc-supervisor  = {某某教授},
    % assoc-supervisor* = {Prof. Uom Uom},
    %
    % 学号
    %
    id              = {21080140314},
    %
    % 学位
    %   本科生不需要填写
    %
    % zh / degree          = {工学硕士},
    % en / degree          = {Master of Engineering},
    %
    % 专业
    %
    zh / major           = {材料科学与工程专业},
    en / major           = {A Very Important Major},
    %
    % 所属院系
    %
    zh / department      = {材料成型及控制工程},
    en / department      = {Depart of XXX},
    %
    % 答辩日期
    %   使用 ISO 格式 (yyyy-mm-dd);默认为当前时间
    %
    % date                 = {2023-05-18},
    %
    % 标题页显示日期
    %   覆盖对应标题页的日期显示,原样输出
    %
    % zh / display-date    = {2023 年 5 月},
    %
    % 资助基金
    %
    % zh / fund  = {
    %                {国家 973 项目 (No. 2025CB000000)},
    %                {国家自然科学基金 (No. 81120250000)},
    %              },
    % en / fund  = {
    %                {National Basic Research Program of China (Grant No. 2025CB000000)},
    %                {National Natural Science Foundation of China (Grant No. 81120250000)},
    %              },
  },
  %
  % 风格设置
  %
  style = {%
  % title-logo-color = black,
      % 论文标题页 logo 颜色 (red/blue/black)
},
%%%%%%%%%%%%%%%%%%%%%%%%%%%%%%%%%%%%%%%%%%%%++++%+++++++
    %
    %%    注::::如果想要该第一页面校徽logo请直接更换
    %             texmf\tex\latex\sjtutex\vi\sjtu-vi-badge-red.pdf      %默认宽度7cm  
    %
    %      texmf\tex\latex\sjtutex\sjtu-lang-thesis-zh.def
    %         %代码如下 
  %  { logo    }
  %     {
  %       content     =
  %         {
  %           \includegraphics [ width = 7 cm ]
  %             { sjtu-vi-badge- \l__sjtu_style_title_logo_color_tl .pdf }
  %         }
  %     },
%%%%%%%%%%%%%%%%%%%%%%%%%%%%%%%%%%%%%%%%%%%%%%%%%%%%%%%%%%%%%????%%%%%%%%%%%%%%%%%%%%%%%%%%%
  %
  % 名称设置
  %
  name = {
    % bib             = {References},
    % ack             = {谢\hspace{\ccwd}辞},
    % achv            = {攻读学位期间完成的论文},
  },
}
% 使用 BibLaTeX 处理参考文献
%   biblatex-gb7714-2015 常用选项
%     gbnamefmt=lowercase     姓名大小写由输入信息确定
%     gbpub=false             禁用出版信息缺失处理
\usepackage[backend=biber,style=gb7714-2015]{biblatex}
% 文献表字体
% \renewcommand{\bibfont}{\zihao{5}\fixedlineskip{15.6bp}}
% 文献表条目间的间距
\setlength{\bibitemsep}{0pt}
% 导入参考文献数据库
\addbibresource{refs.bib}

% 脚注格式
\usepackage[perpage,bottom,hang]{footmisc}

% 定义图片文件目录与扩展名
\graphicspath{{figures/}}
\DeclareGraphicsExtensions{.pdf,.eps,.png,.jpg,.jpeg}

% 确定浮动对象的位置,可以使用 [H],强制将浮动对象放到这里(可能效果很差)
% \usepackage{float}

% 固定宽度的表格
% \usepackage{tabularx}

\usepackage{titlesec}
% 主要涉及 titlesec 宏包的使用。章节样式调整使用\titlelabel,
% \titleformat*命令。前者需要配合计数器使用,后者简单地设置章节标题的


% 使用三线表:toprule,midrule,bottomrule。
\usepackage{booktabs}

% 表格中支持跨行
\usepackage{multirow}

% 表格中数字按小数点对齐
\usepackage{dcolumn}
\newcolumntype{d}[1]{D{.}{.}{#1}}

% 使用长表格
\usepackage{longtable}

% 附带脚注的表格
\usepackage{threeparttable}

% 附带脚注的长表格
\usepackage{threeparttablex}

% 算法环境宏包
\usepackage[ruled,vlined,linesnumbered]{algorithm2e}
% \usepackage{algorithm, algorithmicx, algpseudocode}

% 代码环境宏包
\usepackage{listings}
\lstdefinestyle{lstStyleCode}{%
  aboveskip         = \medskipamount,
  belowskip         = \medskipamount,
  basicstyle        = \ttfamily\zihao{6},
  commentstyle      = \slshape\color{black!60},
  stringstyle       = \color{green!40!black!100},
  keywordstyle      = \bfseries\color{blue!50!black},
  extendedchars     = false,
  upquote           = true,
  tabsize           = 2,
  showstringspaces  = false,
  xleftmargin       = 1em,
  xrightmargin      = 1em,
  breaklines        = false,
  framexleftmargin  = 1em,
  framexrightmargin = 1em,
  backgroundcolor   = \color{gray!10},
  columns           = flexible,
  keepspaces        = true,
  texcl             = true,
  mathescape        = true
}
\lstnewenvironment{codeblock}[1][]{%
  \lstset{style=lstStyleCode,#1}}{}

% 直立体数学符号
\providecommand{\dd}{\mathop{}\!\mathrm{d}}
\providecommand{\ee}{\mathrm{e}}
\providecommand{\ii}{\mathrm{i}}
\providecommand{\jj}{\mathrm{j}}

% 国际单位制宏包
\usepackage{siunitx}

% 定理环境宏包
\usepackage{ntheorem}
% \usepackage{amsthm}

% 绘图宏包
\usepackage{tikz}
\usetikzlibrary{arrows.meta, shapes.geometric}

% 数据图表宏包
\usepackage{pgfplots}
\pgfplotsset{compat=newest}

% 一些文档中用到的 logo
\usepackage{hologo}
\providecommand{\XeTeX}{\hologo{XeTeX}}
\providecommand{\BibLaTeX}{\textsc{Bib}\LaTeX}

% 借用 ltxdoc 里面的几个命令方便写文档
\DeclareRobustCommand\cs[1]{\texttt{\char`\\#1}}
\providecommand\pkg[1]{{\sffamily#1}}

% hyperref 宏包在最后调用
\usepackage{hyperref}

% E-mail
\providecommand{\email}[1]{\href{mailto:#1}{\urlstyle{tt}\nolinkurl{#1}}}


\begin{document}

%TC:ignore

% 标题页
\maketitle

% 原创性声明及使用授权书
\copyrightpage
% 插入外置原创性声明及使用授权书
% 此时必须在导言区使用 \usepackage{pdfpages}
% \copyrightpage[scans/sample-copyright.pdf]

% 前置部分
\frontmatter

% 摘要
% !TEX root = ../main.tex

\begin{abstract}[zh]
  中文摘要应该将学位论文的内容要点简短明了地表达出来,应该包含论文中的基本信息,
  体现科研工作的核心思想。摘要内容应涉及本项科研工作的目的和意义、研究方法、研究
  成果、结论及意义。注意the big tree突出学位论文中具有创新性的成果和新见解的部分。摘要中不宜
  使用公式、化学结构式、图表和非公知公用的符号和术语,不标注引用文献编号。硕士学
  位论文中文摘要字数为 500 字左右,博士学位论文中文摘要字数为 800 字左右。英文摘
  要内容应与中文摘要内容一致。

  摘要页的下方注明本文的关键词(4 \textasciitilde{} 6个)。
\end{abstract}

\begin{abstract}[en]
  Shanghai Jiao Tong University (SJTU) is a key university in China. SJTU was
  founded in 1896. It is one of the oldest universities in China. The University
  has nurtured large numbers of outstanding figures include JIANG Zemin, DING
  Guangen, QIAN Xuesen, Wu Wenjun, WANG An, etc.

  SJTU has beautiful campuses, Bao Zhaolong Library, Various laboratories. It
  has been actively involved in international academic exchange programs. It is
  the center of CERNet in east China region, through computer networks, SJTU has
  faster and closer connection with the world.
\end{abstract}


% 目录
\tableofcontents
% 插图索引
\listoffigures*
% 表格索引
\listoftables*
% 算法索引
\listofalgorithms*

% 符号对照表
\input{contents/nomenclature}

%TC:endignore

% 主体部分
\mainmatter

% 正文内容
% !TEX root = ../main.tex
\chapter{简介we}
\section{二级标题}
\subsection{三级标题}
{\kaishu 相传,在古时候,有个名叫万年的青年,看到当时节令很乱,就有了想把节令定准的打算。但是苦于找不到计算时间的方法,一天,他上山砍柴累了,坐在树阴下休息,树影的移动启发了他,他设计了一个测日影计天时的晷仪,测定一天的时间,后来,山崖上的滴泉启发了他的灵感,他又动手做了一个五层漏壶,来计算时间。天长日久,他发现每隔三百六十多天,四季就轮回一次,天时的长短就重复一遍。
当时的国君叫祖乙,也常为天气风云的不测感到苦恼。万年知道后,就带着日晷和漏壶去见皇上,对祖乙讲清了日月运行的道理。祖乙听后龙颜大悦,感到有道理。于是把万年留下,在天坛前修建日月阁,筑起日晷台和漏壶亭。并希望能测准日月规律,推算出准确的晨夕时间,创建历法,为天下的黎民百姓造福。
有一次,祖乙去了解万年测试历法的进展情况。当他登上日月坛时,看见天坛边的石壁上刻着一首诗:
日出日落三百六,周而复始从头来。
草木枯荣分四时,一岁月有十二圆。
知道万年创建历法已成,亲自登上日月阁看望万年。万年指着天象,对祖乙说:“现在正是十二个月满,旧岁已完,新春复始,祈请国君定个节吧”。祖乙说:“春为岁首,就叫春节吧”。据说这就是春节的来历。
冬去春来,年复一年,万年经过长期观察,精心推算,制定出了准确的太阳历,当他把太阳历呈奉给继任的国君时,已是满面银须。国君深为感动,为纪念万年的功绩,便将太阳历命名为“万年历

}
% !TEX root = ../main.tex

\chapter{简介}
% \LaTeX 中,将文档分为若干大纲级别。分别是:
% \part 部分。这个大纲不会打断 chapter 的编号。
% \chapter 章。基于 article 的文档类不含该大纲级别。
% \section 节。
% \subsection 次节。默认 report/book 文档类本级别及以下的大纲不进行
% 编号,也不纳入目录。
这是 SJTUThesis 的示例文档,基本上覆盖了模板中所有格式的设置。建议大家在使用模
板之前,除了阅读《SJTUThesis 使用文档》,这个示例文档也最好能看一看。

\section{二级标题}

\subsection{三级标题}

\subsubsection{四级标题}
% \titleformat{command}[shape]{format}{label}{sep}{before-code}[after-code]
% 它们对应的含义如下:
% command: 大纲级别命令,如\chapter等。
% shape: 章节的预定义样式,分为 9 种:
    % hang 缺省值。标题在右侧,紧跟在标签后。
    % block 标题和标签封装排版,不允许额外的格式控制。
    % display 标题另起一段,位于标签的下方。
    % runin 标题与标签同行,且正文从标题右侧开始。
    % leftmargin 标题和标签分段,且位于左页边。
    % rightmargin 仿上。右页边。
    % drop 文本包围标题。
    % wrap 类似 drop,文本会自动调整以适应最长的一行。
    % frame 类似 display,但有框线。
% format: 用于设置标签和标题文字的字体样式。这里可以包含竖直空距,即
% 标题文字到正文的距离。
% label: 用于设置标签的样式,比如“第\chinese\thechapter章”大概是
% ctexbook 类的默认样式设置。
% sep: 标签和标题文字的水平间距,必须是 L A TEX 的长度表达。当 shape 取
% display 时,表示竖直空距;取 frame 时表示标题到文本框的距离。
% before: 标题前的内容。
% after: 标题后的内容。对于 hang, block, display,此内容取竖向;对于 runin,
% leftmargin, 此内容取横向;否则此内容被忽略。
Lorem ipsum dolor sit amet, consectetur adipiscing elit, sed do eiusmod tempor

\section{脚注}

Lorem ipsum dolor sit amet, consectetur adipiscing elit, sed do eiusmod tempor
incididunt ut labore\footnote{第一个} et dolore magna aliqua. \footnote{%这里添加脚注
Ut第二个}

\section{字体}
% 中文字体
{\songti 宋体} {\heiti 黑体} {\kaishu 楷体} {\fangsong 仿宋}

中文字体的\textbf{粗体}与\textit{斜体}

% 字体大小
{\tiny          Hello}\\
{\scriptsize    Hello}\\
{\footnotesize  Hello}\\
{\small         Hello}\\
{\normalsize    Hello}\\
{\large         Hello}\\
{\Large         Hello}\\
{\LARGE         Hello}\\
{\huge          Hello}\\
{\Huge          Hello}

% 中文字号设置
\zihao{0} 你好!\\
\zihao{-1} 你好!\\
\zihao{-2} 你好!\\
\zihao{1} 你好!\\
\zihao{2} 你好!\\
\zihao{3} 你好!\\

{\songti \zihao{3} \textbf{九~~~世纪}\\十九~~~世~\textit{斜体}~纪胆推进改革:率先组成教授
}
{\songti%可以改字体
 改革开放以来,学校以“敢为天下先”的精神,大胆推进改革:率先组成教授代
}

{\ifcsname fangsong\endcsname\fangsong\else[无 \cs{fangsong} 字体。]\fi 交通大学
  始终把人才培养作为办学的根本任务。一百多年来,学校为国家和社会培养了 20余万各
}

{\ifcsname kaishu\endcsname\kaishu\else[无 \cs{kaishu} 字体。]\fi 截至 2011 年 12
  月 31 日,学校共有 24 个学院 / 直属系(另有继续教育学院、技术学院和国际教育学
 }
% !TEX root = ../main.tex

\chapter{数学与引用文献的标注}

\section{数学}

\subsection{数字和单位}

宏包 \pkg{siunitx} 提供了更好的数字和单位支持:
% \pkg{}这段代码用于调用 LaTeX 中的 siunitx 宏包,siunitx 宏包主要用于排版物理量和单位。下面是一些详细的解释:

% \pkg{siunitx}: 这里 \pkg{} 是一个命令,用于在文档中标记和显示宏包的名称。在这里,\pkg{siunitx} 表示引用并调用 siunitx 宏包。

% siunitx 宏包提供了一些命令和选项,用于排版数学和科学文桌中的物理量和单位。例如,可以使用 \SI{数值}{单位} 命令来排版物理量和单位,还可以设置数字格式、小数点显示等。

% 如果你需要在 LaTeX 中排版物理量和单位,并且希望有更好的排版效果和灵活性,可以考虑使用 siunitx 宏包。

% 希望这个解释能够帮助你理解 \pkg{siunitx} 这段代码的含义。如果有任何疑问,请随时向我提问。
\begin{itemize}
  \item \num{12345.67890}
  \item \complexnum{1+-2i}
  \item \num{.3e45}
  \item \numproduct{1.654 x 2.34 x 3.430}
  \item \unit{kg.m.s^{-1}}
  \item \unit{\micro\meter} $\unit{\micro\meter}$
  \item \unit{\ohm} $\unit{\ohm}$
  \item \numlist{10;20}
  \item \numlist{10;20;30}
  \item \qtylist{0.13;0.67;0.80}{\milli\metre}
  \item \numrange{10}{20}
  \item \qtyrange{10}{20}{\degreeCelsius}
\end{itemize}

\subsection{数学符号和公式}

按照国标 GB/T 3102.11—1993《物理科学和技术中使用的数学符号》,
微分符号 $\dd$ 应使用直立体。除此之外,数学常数也应使用直立体:
\begin{itemize}
  \item 微分符号 $\dd$:\cs{dd}
  \item 圆周率 $\uppi$:\cs{uppi}
  \item 自然对数的底 $\ee$:\cs{ee}
  \item 虚数单位 $\ii$, $\jj$:\cs{ii} \cs{jj}
\end{itemize}

公式应另起一行居中排版。公式后应注明编号,按章顺序编排,编号右端对齐。
\begin{equation}
  \ee^{\ii\uppi} + 1 = 0,
\end{equation}
\begin{equation}
  \frac{\dd^2 u}{\dd t^2} = \int f(x) \dd x.
\end{equation}

公式末尾是需要添加标点符号的,至于用逗号还是句号,取决于公式下面一句是接着公式说的,还是另起一句。
\begin{equation}
  \frac{2h}{\pi}\int_{0}^{\infty}\frac{\sin\left( \omega\delta \right)}{\omega}
  \cos\left( \omega x \right) \dd\omega = 
  \begin{cases}
    h, & \left| x \right| < \delta, \\
    \frac{h}{2}, & x = \pm \delta, \\
    0, & \left| x \right| > \delta.
  \end{cases}
\end{equation}
公式较长时最好在等号“$=$”处转行。
\begin{align}
    & I (X_3; X_4) - I (X_3; X_4 \mid X_1) - I (X_3; X_4 \mid X_2) \nonumber \\
  = & [I (X_3; X_4) - I (X_3; X_4 \mid X_1)] - I (X_3; X_4 \mid \tilde{X}_2) \\
  = & I (X_1; X_3; X_4) - I (X_3; X_4 \mid \tilde{X}_2).
\end{align}

如果在等号处转行难以实现,也可在 $+$、$-$、$\times$、$\div$ 运算符号处转行,转行
时运算符号仅书写于转行式前,不重复书写。
\begin{multline}
  \frac{1}{2} \Delta (f_{ij} f^{ij}) =
    2 \left(\sum_{i<j} \chi_{ij}(\sigma_{i} - \sigma_{j})^{2}
    + f^{ij} \nabla_{j} \nabla_{i} (\Delta f) \right. \\
  \left. + \nabla_{k} f_{ij} \nabla^{k} f^{ij} +
    f^{ij} f^{k} \left[2\nabla_{i}R_{jk}
    - \nabla_{k} R_{ij} \right] \vphantom{\sum_{i<j}} \right).
\end{multline}

\subsection{定理环境}

示例文件中使用 \pkg{ntheorem} 宏包配置了定理、引理和证明等环境。用户也可以使用
\pkg{amsthm} 宏包。

这里举一个“定理”和“证明”的例子。
\begin{theorem}[留数定理]
\label{thm:res}
  假设 $U$ 是复平面上的一个单连通开子集,$a_1, \ldots, a_n$ 是复平面上有限个点,
  $f$ 是定义在 $U \backslash \{a_1, \ldots, a_n\}$ 上的全纯函数,如果 $\gamma$
  是一条把 $a_1, \ldots, a_n$ 包围起来的可求长曲线,但不经过任何一个 $a_k$,并且
  其起点与终点重合,那么:

  \begin{equation}
    \label{eq:res}
    \oint\limits_\gamma f(z)\, \dd z = 2\uppi \ii \sum_{k=1}^n \operatorname{I}(\gamma, a_k) \operatorname{Res}(f, a_k).
  \end{equation}

  如果 $\gamma$ 是若尔当曲线,那么 $\operatorname{I}(\gamma, a_k) = 1$,因此:

  \begin{equation}
    \label{eq:resthm}
    \oint\limits_\gamma f(z)\, \dd z = 2\uppi \ii \sum_{k=1}^n \operatorname{Res}(f, a_k).
  \end{equation}

  在这里,$\operatorname{Res}(f, a_k)$ 表示 $f$ 在点 $a_k$ 的留数,
  $\operatorname{I}(\gamma, a_k)$ 表示 $\gamma$ 关于点 $a_k$ 的卷绕数。卷绕数是
  一个整数,它描述了曲线 $\gamma$ 绕过点 $a_k$ 的次数。如果 $\gamma$ 依逆时针方
  向绕着 $a_k$ 移动,卷绕数就是一个正数,如果 $\gamma$ 根本不绕过 $a_k$,卷绕数
  就是零。

  定理~\ref{thm:res} 的证明。

  \begin{proof}
    首先,由……

    其次,……

    所以……
  \end{proof}
\end{theorem}

\section{引用文献的标注}

按照教务处的要求,参考文献外观应符合国标 GB/T 7714 的要求。模版使用 \BibLaTeX{}
配合 \pkg{biblatex-gb7714-2015} 样式包%
\footnote{\url{https://www.ctan.org/pkg/biblatex-gb7714-2015}}%
控制参考文献的输出样式,后端采用 \pkg{biber} 管理文献。

请注意 \pkg{biblatex-gb7714-2015} 宏包 2016 年 9 月才加入 CTAN,如果你使用的
\TeX{} 系统版本较旧,可能没有包含 \pkg{biblatex-gb7714-2015} 宏包,需要手动安装。
\BibLaTeX{} 与 \pkg{biblatex-gb7714-2015} 目前在活跃地更新,为避免一些兼容性问
题,推荐使用较新的版本。

正文中引用参考文献时,使用 \verb|\cite{key1,key2,key3...}| 可以产生“上标引用的参
考文献”,如 \cite{Yu2001,Cheng1999,LSC1957}。使用
\verb|\parencite{key1,key2,key3...}| 则可以产生水平引用的参考文献,例如
\parencite{Li1999,Jiang1989,Hopkinson1999}。请看下面的例子,将会穿插使用水平的和
上标的参考文献:普通图书\parencite{Yu2001,Jiang1998},论文集、会议录
\cite{CSTAM1990},科技报告\parencite{WHO1970},学位论文\cite{Zhang1998},专利文
献\parencite{Jiang1989,HBLZ2001},专著中析出的文献\cite{Cheng1999,GBT2659},期刊
中析出的文献\parencite{Li1999,Li2000},报纸中析出的文献\cite{Ding2000}, 电子文献
\parencite{Jiang1999,Christine1998,Xiao2001}。

可以使用 \verb|\nocite{key1,key2,key3...}| 将参考文献条目加入到文献表中但不在正
文中引用。使用 \verb|\nocite{*}| 可以将参考文献数据库中的所有条目加入到文献表
中。
\nocite{Yang1999,Schinstock2000,Wen1990,GBT16159}
\input{contents/floats}
\input{contents/summary}
% \input{contents/digest.tex}
% 可以使用\input添加章节树
%TC:ignore

% 参考文献
\printbibliography[heading=bibintoc]

% 附录
\appendix

% 附录中图表不加入索引
\captionsetup{list=no}

% 附录内容
\input{contents/app_maxwell_equations}
\input{contents/app_flow_chart}

% 结尾部分
\backmatter

% 用于盲审的论文需隐去致谢、发表论文、科研成果、简历

% 致谢
\input{contents/acknowledgements}

% 发表论文及科研成果
% 盲审论文中,发表论文及科研成果等仅以第几作者注明即可,不要出现作者或他人姓名
\input{contents/achievements}

% 简历
\input{contents/resume}

% 学士学位论文要求在最后有一个大摘要,单独编页码
\input{contents/digest}

%TC:endignore

\end{document}
