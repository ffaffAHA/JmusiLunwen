% !TEX root = ../main.tex
\chapter{简介we}
\section{二级标题}
\subsection{三级标题}
{\kaishu 相传,在古时候,有个名叫万年的青年,看到当时节令很乱,就有了想把节令定准的打算。但是苦于找不到计算时间的方法,一天,他上山砍柴累了,坐在树阴下休息,树影的移动启发了他,他设计了一个测日影计天时的晷仪,测定一天的时间,后来,山崖上的滴泉启发了他的灵感,他又动手做了一个五层漏壶,来计算时间。天长日久,他发现每隔三百六十多天,四季就轮回一次,天时的长短就重复一遍。
当时的国君叫祖乙,也常为天气风云的不测感到苦恼。万年知道后,就带着日晷和漏壶去见皇上,对祖乙讲清了日月运行的道理。祖乙听后龙颜大悦,感到有道理。于是把万年留下,在天坛前修建日月阁,筑起日晷台和漏壶亭。并希望能测准日月规律,推算出准确的晨夕时间,创建历法,为天下的黎民百姓造福。
有一次,祖乙去了解万年测试历法的进展情况。当他登上日月坛时,看见天坛边的石壁上刻着一首诗:
日出日落三百六,周而复始从头来。
草木枯荣分四时,一岁月有十二圆。
知道万年创建历法已成,亲自登上日月阁看望万年。万年指着天象,对祖乙说:“现在正是十二个月满,旧岁已完,新春复始,祈请国君定个节吧”。祖乙说:“春为岁首,就叫春节吧”。据说这就是春节的来历。
冬去春来,年复一年,万年经过长期观察,精心推算,制定出了准确的太阳历,当他把太阳历呈奉给继任的国君时,已是满面银须。国君深为感动,为纪念万年的功绩,便将太阳历命名为“万年历

}