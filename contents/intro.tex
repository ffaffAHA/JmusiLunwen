% !TEX root = ../main.tex

\chapter{简介}
% \LaTeX 中,将文档分为若干大纲级别。分别是:
% \part 部分。这个大纲不会打断 chapter 的编号。
% \chapter 章。基于 article 的文档类不含该大纲级别。
% \section 节。
% \subsection 次节。默认 report/book 文档类本级别及以下的大纲不进行
% 编号,也不纳入目录。
这是 SJTUThesis 的示例文档,基本上覆盖了模板中所有格式的设置。建议大家在使用模
板之前,除了阅读《SJTUThesis 使用文档》,这个示例文档也最好能看一看。

\section{二级标题}

\subsection{三级标题}

\subsubsection{四级标题}
% \titleformat{command}[shape]{format}{label}{sep}{before-code}[after-code]
% 它们对应的含义如下:
% command: 大纲级别命令,如\chapter等。
% shape: 章节的预定义样式,分为 9 种:
    % hang 缺省值。标题在右侧,紧跟在标签后。
    % block 标题和标签封装排版,不允许额外的格式控制。
    % display 标题另起一段,位于标签的下方。
    % runin 标题与标签同行,且正文从标题右侧开始。
    % leftmargin 标题和标签分段,且位于左页边。
    % rightmargin 仿上。右页边。
    % drop 文本包围标题。
    % wrap 类似 drop,文本会自动调整以适应最长的一行。
    % frame 类似 display,但有框线。
% format: 用于设置标签和标题文字的字体样式。这里可以包含竖直空距,即
% 标题文字到正文的距离。
% label: 用于设置标签的样式,比如“第\chinese\thechapter章”大概是
% ctexbook 类的默认样式设置。
% sep: 标签和标题文字的水平间距,必须是 L A TEX 的长度表达。当 shape 取
% display 时,表示竖直空距;取 frame 时表示标题到文本框的距离。
% before: 标题前的内容。
% after: 标题后的内容。对于 hang, block, display,此内容取竖向;对于 runin,
% leftmargin, 此内容取横向;否则此内容被忽略。
Lorem ipsum dolor sit amet, consectetur adipiscing elit, sed do eiusmod tempor

\section{脚注}

Lorem ipsum dolor sit amet, consectetur adipiscing elit, sed do eiusmod tempor
incididunt ut labore\footnote{第一个} et dolore magna aliqua. \footnote{%这里添加脚注
Ut第二个}

\section{字体}
% 中文字体
{\songti 宋体} {\heiti 黑体} {\kaishu 楷体} {\fangsong 仿宋}

中文字体的\textbf{粗体}与\textit{斜体}

% 字体大小
{\tiny          Hello}\\
{\scriptsize    Hello}\\
{\footnotesize  Hello}\\
{\small         Hello}\\
{\normalsize    Hello}\\
{\large         Hello}\\
{\Large         Hello}\\
{\LARGE         Hello}\\
{\huge          Hello}\\
{\Huge          Hello}

% 中文字号设置
\zihao{0} 你好!\\
\zihao{-1} 你好!\\
\zihao{-2} 你好!\\
\zihao{1} 你好!\\
\zihao{2} 你好!\\
\zihao{3} 你好!\\

{\songti \zihao{3} \textbf{九~~~世纪}\\十九~~~世~\textit{斜体}~纪胆推进改革:率先组成教授
}
{\songti%可以改字体
 改革开放以来,学校以“敢为天下先”的精神,大胆推进改革:率先组成教授代
}

{\ifcsname fangsong\endcsname\fangsong\else[无 \cs{fangsong} 字体。]\fi 交通大学
  始终把人才培养作为办学的根本任务。一百多年来,学校为国家和社会培养了 20余万各
}

{\ifcsname kaishu\endcsname\kaishu\else[无 \cs{kaishu} 字体。]\fi 截至 2011 年 12
  月 31 日,学校共有 24 个学院 / 直属系(另有继续教育学院、技术学院和国际教育学
 }